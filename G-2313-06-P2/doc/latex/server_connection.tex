Esta sección incluye las funciones de conexión, inicialización del socket, listen de nuevos clientes, parseo de comandos de I\+R\+C, declaración y manejo del array de punteros a funciones para el parseo de comandos, inicalización del Thread\+Pool para los hilos que se asignarán a los clientes y el proceso de daemonización del servidor.\hypertarget{server_connection_cabeceras}{}\section{Cabeceras}\label{server_connection_cabeceras}
{\ttfamily  {\bfseries \#include} {\bfseries $<$\hyperlink{G-2313-06-P1__server_8h}{G-\/2313-\/06-\/\+P1\+\_\+server.\+h}$>$} } \hypertarget{server_connection_variables}{}\section{Variables definidas}\label{server_connection_variables}
Se utilizan las siguientes variables definidas en \hyperlink{G-2313-06-P1__server_8h}{G-\/2313-\/06-\/\+P1\+\_\+server.\+h} 

{\ttfamily  \#define M\+I\+N\+\_\+\+P\+O\+O\+L\+\_\+\+T\+H\+R\+E\+A\+D\+S 3 ~\newline
\#define M\+A\+X\+\_\+\+C\+O\+N\+N\+E\+C\+T\+I\+O\+N\+S 100 ~\newline
\#define S\+E\+R\+V\+E\+R\+\_\+\+P\+O\+R\+T 6667 ~\newline
\#define C\+L\+I\+E\+N\+T\+\_\+\+M\+E\+S\+S\+A\+G\+E\+\_\+\+M\+A\+X\+S\+I\+Z\+E 8096 }\hypertarget{server_connection_functions}{}\section{Funciones implementadas}\label{server_connection_functions}
Se incluyen las siguientes funciones\+: 
\begin{DoxyItemize}
\item \hyperlink{server_start}{server\+\_\+start} 
\item \hyperlink{server_accept_connection}{server\+\_\+accept\+\_\+connection} 
\item \hyperlink{server_start_communication}{server\+\_\+start\+\_\+communication} 
\item \hyperlink{server_execute_function}{server\+\_\+execute\+\_\+function} 
\item \hyperlink{server_exit}{server\+\_\+exit} 
\item \hyperlink{server_check_socket_status}{server\+\_\+check\+\_\+socket\+\_\+status} 
\item \hyperlink{server_daemon}{server\+\_\+daemon} 
\item \hyperlink{server_start_pool}{server\+\_\+start\+\_\+pool} 
\end{DoxyItemize}\hypertarget{server_start}{}\section{server\+\_\+start}\label{server_start}
Inicia el servidor (T\+C\+P)\hypertarget{server_start_synopsis2}{}\subsection{Synopsis}\label{server_start_synopsis2}
{\ttfamily  {\bfseries \#include} {\bfseries $<$\hyperlink{G-2313-06-P1__server_8h}{G-\/2313-\/06-\/\+P1\+\_\+server.\+h}$>$} ~\newline
 {\bfseries int} {\bfseries server\+\_\+start} {\bfseries }() } \hypertarget{server_start_descripcion2}{}\subsection{Descripción}\label{server_start_descripcion2}
Inicializa el socket principal del servidor, para ello se genera una llamada a {\bfseries socket()} para iniciar el socket. A dicha función se le pasan como parámetros valores adecuados para la inicialización\+: 
\begin{DoxyItemize}
\item A\+F\+\_\+\+I\+N\+E\+T 
\item S\+O\+C\+K\+\_\+\+S\+T\+R\+E\+A\+M 
\end{DoxyItemize}Tras generar el identificador del socket con la llamada anterior, se asignan parámetros adicionales (como S\+O\+L\+\_\+\+S\+O\+C\+K\+E\+T y S\+O\+\_\+\+R\+E\+U\+S\+E\+A\+D\+D\+R), además se genera la estructura sockaddr\+\_\+in de dirección, en la que se especifican ciertos parámetros\+: 
\begin{DoxyItemize}
\item {\bfseries sin\+\_\+family\+:} Se indica que la conexión es T\+C\+P/\+I\+P. 
\item {\bfseries sin\+\_\+port\+:} Se indica que el puerto por defecto es 6667. 
\item {\bfseries sin\+\_\+addr\+:} Se indica que cualquier I\+P es admitida al conectar (I\+N\+A\+D\+D\+R\+\_\+\+A\+N\+Y). 
\end{DoxyItemize}Tras generar la configuración básica del socket, se utilizan las funciones bind() y listen() para finalizar con la conexión del socket principal al que accederán los clientes mediante el puerto por defecto\+: 
\begin{DoxyItemize}
\item {\bfseries bind()\+:} Realiza una unión entre el identificador del socket generado y el struct de Direccion mencionado anteriormente para asignar los parametros del struct a ese socket. 
\item {\bfseries listen()\+:} Indica al socket que debe admitir conexiones en el identificador del socket generado anteriormente y además le pasa el parámetros de conexiones máximas que en este caso está fijado por M\+A\+X\+\_\+\+C\+O\+N\+N\+E\+C\+T\+I\+O\+N\+S (100). 
\end{DoxyItemize}\hypertarget{server_start_return2}{}\subsection{Valores devueltos}\label{server_start_return2}

\begin{DoxyItemize}
\item {\bfseries int} Descriptor del master socket generado para atender las peticiones. 
\end{DoxyItemize}\hypertarget{server_start_authors2}{}\subsection{Autores}\label{server_start_authors2}

\begin{DoxyItemize}
\item Jorge Parrilla Llamas (\href{mailto:jorge.parrilla@estudiante.uam.es}{\tt jorge.\+parrilla@estudiante.\+uam.\+es}) 
\item Javier de Marco Tomás (\href{mailto:javier.marco@estudiante.uam.es}{\tt javier.\+marco@estudiante.\+uam.\+es}) 
\end{DoxyItemize}\hypertarget{server_accept_connection}{}\section{server\+\_\+accept\+\_\+connection}\label{server_accept_connection}
Acepta las conexiones recibidas en el socket.\hypertarget{server_accept_connection_synopsis3}{}\subsection{Synopsis}\label{server_accept_connection_synopsis3}
{\ttfamily  {\bfseries \#include} {\bfseries $<$\hyperlink{G-2313-06-P1__server_8h}{G-\/2313-\/06-\/\+P1\+\_\+server.\+h}$>$} ~\newline
 {\bfseries void \hyperlink{G-2313-06-P1__server_8c_aaac8642d2e699e0f9d942d28a9b233c2}{server\+\_\+accept\+\_\+connection(int socket\+\_\+desc)}} } \hypertarget{server_accept_connection_descripcion3}{}\subsection{Descripción}\label{server_accept_connection_descripcion3}
Acepta las conexiones recibidas mediante el descriptor {\bfseries socket\+\_\+desc} recibido como parámetro de la función. Inicia un bucle que realiza un {\bfseries accept()} a los clientes recibidos mediante el descriptor del socket. Esta función genera un identificador único para los clientes (un descriptor del socket). En caso encontrar un error, finaliza la ejecución del servidor. 

Tras recibir al cliente y asignarle un descriptor, genera una tarea (task) para dicho cliente utilizando la librería generada para tal efecto (Thread\+Pool). Esta tarea se añade a un hilo de ejecución exclusivo para el cliente.\hypertarget{server_accept_connection_return3}{}\subsection{Valores devueltos}\label{server_accept_connection_return3}

\begin{DoxyItemize}
\item {\bfseries void} No devuelve nada. 
\end{DoxyItemize}\hypertarget{server_accept_connection_authors3}{}\subsection{Autores}\label{server_accept_connection_authors3}

\begin{DoxyItemize}
\item Jorge Parrilla Llamas (\href{mailto:jorge.parrilla@estudiante.uam.es}{\tt jorge.\+parrilla@estudiante.\+uam.\+es}) 
\item Javier de Marco Tomás (\href{mailto:javier.marco@estudiante.uam.es}{\tt javier.\+marco@estudiante.\+uam.\+es}) 
\end{DoxyItemize}\hypertarget{server_start_communication}{}\section{server\+\_\+start\+\_\+communication}\label{server_start_communication}
Ejecuta el control (main) principal para cada hilo.\hypertarget{server_start_communication_synopsis4}{}\subsection{Synopsis}\label{server_start_communication_synopsis4}
{\ttfamily  {\bfseries \#include} {\bfseries $<$\hyperlink{G-2313-06-P1__server_8h}{G-\/2313-\/06-\/\+P1\+\_\+server.\+h}$>$} ~\newline
 {\bfseries void $\ast$server\+\_\+start\+\_\+communication(int socket\+\_\+desc)} } \hypertarget{server_start_communication_descripcion4}{}\subsection{Descripción}\label{server_start_communication_descripcion4}
Contiene el código que ejecutan los hijos al recibir la conexión de un cliente en {\bfseries \hyperlink{G-2313-06-P1__server_8c_aaac8642d2e699e0f9d942d28a9b233c2}{server\+\_\+accept\+\_\+connection()}}. Declara varias variables que se pasan como parámetros al resto de funciones utilizadas para el parseo de comandos\+:


\begin{DoxyItemize}
\item {\bfseries char $\ast$nick\+:} Almacena el nick del personaje, se reserva memoria y se pasa como puntero a las funciones de parseo de comandos en la que la función destinada al comando nick utilizará un {\bfseries strcpy} para actualizar dicho valor. 
\item {\bfseries int $\ast$register\+\_\+status\+:} Entero que representa el estado actual del cliente del hilo, sirve de control para las funciones de parseo y cierre de socket\+: 
\begin{DoxyItemize}
\item 0\+: Indica que no se ha efectuado un registro válido del usuario (comando N\+I\+C\+K y U\+S\+E\+R). 
\item 1\+: Indica que se ha parseado correctamente el nick y es posible realizar el registro del usuario (comando U\+S\+E\+R). 
\item 2\+: Indica que se ha registrado correctamente usuario y es posible realizar cualquier otro comando del I\+R\+C. Sirve de control para no ejecutar otros comandos hasta que el usuario no envíe los comandos válidos (N\+I\+C\+K y U\+S\+E\+R) para el registro. A su vez, esto permitirá que en caso de cerrarse el socket se comprobará que si el valor de esta variable es 2 se utilizará la función del T\+A\+D para eliminar al usuario de la lista. 
\end{DoxyItemize}
\end{DoxyItemize}

Contiene un bucle que comprueba si el socket sigue abierto y parsea los comandos del usuario mediante las funciones apropiadas para ello ({\bfseries I\+R\+C\+\_\+\+Un\+Pipeline\+Commands y I\+R\+C\+\_\+\+Command\+Query}). Dichos comandos parseados envían a la función {\bfseries server\+\_\+execute\+\_\+function} que contiene un array de punteros a función para redirigir cada comando a su función apropiada.

Una vez se detecta que el socket se ha cerrado, se elimina al usuario del T\+A\+D en caso de que {\bfseries register\+\_\+status} tenga el valor adecuado (2) y se liberan los punteros reservados para este hilo.\hypertarget{server_start_communication_return4}{}\subsection{Valores devueltos}\label{server_start_communication_return4}

\begin{DoxyItemize}
\item {\bfseries void} No devuelve nada. 
\end{DoxyItemize}\hypertarget{server_start_communication_authors4}{}\subsection{Autores}\label{server_start_communication_authors4}

\begin{DoxyItemize}
\item Jorge Parrilla Llamas (\href{mailto:jorge.parrilla@estudiante.uam.es}{\tt jorge.\+parrilla@estudiante.\+uam.\+es}) 
\item Javier de Marco Tomás (\href{mailto:javier.marco@estudiante.uam.es}{\tt javier.\+marco@estudiante.\+uam.\+es}) 
\end{DoxyItemize}\hypertarget{server_execute_function}{}\section{server\+\_\+execute\+\_\+function}\label{server_execute_function}
Envía a los comandos a las funciones apropiadas por punteros.\hypertarget{server_execute_function_synopsis5}{}\subsection{Synopsis}\label{server_execute_function_synopsis5}
{\ttfamily  {\bfseries \#include} {\bfseries $<$\hyperlink{G-2313-06-P1__server_8h}{G-\/2313-\/06-\/\+P1\+\_\+server.\+h}$>$} ~\newline
 {\bfseries void \hyperlink{G-2313-06-P1__server_8c_a775161328c3264fb8f96981f7a9c83ae}{server\+\_\+execute\+\_\+function(long function\+Name, char$\ast$ command, int desc, char$\ast$ nick, int$\ast$ register\+\_\+status)}} } \hypertarget{server_execute_function_descripcion5}{}\subsection{Descripción}\label{server_execute_function_descripcion5}
Genera un array de punteros a función para enviar a los comandos a su función de parseo y procesamiento. Utiliza el index del T\+A\+D como índice del array y reserva sitio en el array mediante la variable del T\+A\+D {\bfseries I\+R\+C\+\_\+\+M\+A\+X\+\_\+\+U\+S\+E\+R\+\_\+\+C\+O\+M\+M\+A\+N\+D\+S}\+:

{\ttfamily  Function\+Call\+Back functions\mbox{[}I\+R\+C\+\_\+\+M\+A\+X\+\_\+\+U\+S\+E\+R\+\_\+\+C\+O\+M\+M\+A\+N\+D\+S\mbox{]}; ~\newline
int i; ~\newline
char $\ast$msg; ~\newline
for(i=0; i$<$I\+R\+C\+\_\+\+M\+A\+X\+\_\+\+U\+S\+E\+R\+\_\+\+C\+O\+M\+M\+A\+N\+D\+S; i++)\{ ~\newline
 functions\mbox{[}i\mbox{]}=N\+U\+L\+L; ~\newline
\} ~\newline
functions\mbox{[}N\+I\+C\+K\mbox{]} = \& server\+\_\+command\+\_\+nick; ~\newline
functions\mbox{[}U\+S\+E\+R\mbox{]} = \& server\+\_\+command\+\_\+user; ~\newline
...} 

Una vez generado el array con todos los comandos implementados, envía dicho comando a su función pasándole como parámetro las variables adecuadas para ello\+:


\begin{DoxyItemize}
\item {\bfseries char $\ast$command\+:} String que contiene el comando completo recibido. 
\item {\bfseries int $\ast$desc\+:} Entero que representa el descriptor del socket del cliente. 
\item {\bfseries char $\ast$nick\+:} String que contiene el nick del cliente del hilo. 
\item {\bfseries int $\ast$register\+\_\+status\+:} Entero que representa el estado del registro. 
\end{DoxyItemize}

La llamada a la función se hace de la siguiente manera\+:

{\ttfamily  ($\ast$functions\mbox{[}function\+Name\mbox{]})(command, desc, nick, register\+\_\+status); }\hypertarget{server_execute_function_return5}{}\subsection{Valores devueltos}\label{server_execute_function_return5}

\begin{DoxyItemize}
\item {\bfseries void} No devuelve nada. 
\end{DoxyItemize}\hypertarget{server_execute_function_authors5}{}\subsection{Autores}\label{server_execute_function_authors5}

\begin{DoxyItemize}
\item Jorge Parrilla Llamas (\href{mailto:jorge.parrilla@estudiante.uam.es}{\tt jorge.\+parrilla@estudiante.\+uam.\+es}) 
\item Javier de Marco Tomás (\href{mailto:javier.marco@estudiante.uam.es}{\tt javier.\+marco@estudiante.\+uam.\+es}) 
\end{DoxyItemize}\hypertarget{server_exit}{}\section{server\+\_\+exit}\label{server_exit}
Finaliza el servidor I\+R\+C y cierra el socket.\hypertarget{server_exit_synopsis6}{}\subsection{Synopsis}\label{server_exit_synopsis6}
{\ttfamily  {\bfseries \#include} {\bfseries $<$\hyperlink{G-2313-06-P1__server_8h}{G-\/2313-\/06-\/\+P1\+\_\+server.\+h}$>$} ~\newline
 {\bfseries void \hyperlink{G-2313-06-P1__server_8c_a0e947005d451a8f3bf3af01f54b59f11}{server\+\_\+exit()}} } \hypertarget{server_exit_descripcion6}{}\subsection{Descripción}\label{server_exit_descripcion6}
Una vez se finaliza el servidor (se recibe la señal adecuada para ello) se procede a finalizar el servicio actual. Para ello se finaliza el bucle de {\bfseries \hyperlink{G-2313-06-P1__server_8c_aaac8642d2e699e0f9d942d28a9b233c2}{server\+\_\+accept\+\_\+connection()}} actualizando la variable global {\bfseries server\+\_\+status=false} que provoca una interrupción en el while que acepta las conexiones de los clientes.

Además, se finaliza el socket con la función {\bfseries close(server\+\_\+socket\+\_\+desc)}, tras esto se realiza una llamada la función {\bfseries thread\+\_\+pool\+\_\+delete(pool)} para eliminar todos los hilos generados en el Thread\+Pool y se liberan los recursos para proceder a la salida del proceso general.\hypertarget{server_exit_return6}{}\subsection{Valores devueltos}\label{server_exit_return6}

\begin{DoxyItemize}
\item {\bfseries void} No devuelve nada. 
\end{DoxyItemize}\hypertarget{server_exit_authors6}{}\subsection{Autores}\label{server_exit_authors6}

\begin{DoxyItemize}
\item Jorge Parrilla Llamas (\href{mailto:jorge.parrilla@estudiante.uam.es}{\tt jorge.\+parrilla@estudiante.\+uam.\+es}) 
\item Javier de Marco Tomás (\href{mailto:javier.marco@estudiante.uam.es}{\tt javier.\+marco@estudiante.\+uam.\+es}) 
\end{DoxyItemize}\hypertarget{server_check_socket_status}{}\section{server\+\_\+check\+\_\+socket\+\_\+status}\label{server_check_socket_status}
Comprueba si el socket del cliente sigue abierto.\hypertarget{server_check_socket_status_Synopsis}{}\subsection{Synopsis}\label{server_check_socket_status_Synopsis}
{\ttfamily  {\bfseries \#include} {\bfseries $<$\hyperlink{G-2313-06-P1__server_8h}{G-\/2313-\/06-\/\+P1\+\_\+server.\+h}$>$} ~\newline
 {\bfseries int \hyperlink{G-2313-06-P1__server_8c_a64f1fffc5903ccf0350845cd21a95b6e}{server\+\_\+check\+\_\+socket\+\_\+status(int socket\+\_\+desc)}} } \hypertarget{server_check_socket_status_descripcion}{}\subsection{Descripción}\label{server_check_socket_status_descripcion}
Es una función simple que comprueba mediante la función {\bfseries select()} si el socket del cliente recibido como parámetro {\bfseries socket\+\_\+desc} sigue abierto o no. Para ello se hace una selección de dicho socket y si devuelve error o salta el timeout se indica que el socket está cerrado y en la función {\bfseries server\+\_\+start\+\_\+communication} se liberan los recursos necesarios.\hypertarget{server_check_socket_status_return}{}\subsection{Valores devueltos}\label{server_check_socket_status_return}

\begin{DoxyItemize}
\item {\bfseries int} Devuelve el valor verdadero/falso sobre el estado del socket. 
\end{DoxyItemize}\hypertarget{server_check_socket_status_authors}{}\subsection{Autores}\label{server_check_socket_status_authors}

\begin{DoxyItemize}
\item Jorge Parrilla Llamas (\href{mailto:jorge.parrilla@estudiante.uam.es}{\tt jorge.\+parrilla@estudiante.\+uam.\+es}) 
\item Javier de Marco Tomás (\href{mailto:javier.marco@estudiante.uam.es}{\tt javier.\+marco@estudiante.\+uam.\+es}) 
\end{DoxyItemize}\hypertarget{server_daemon}{}\section{server\+\_\+daemon}\label{server_daemon}
Daemoniza el servidor.\hypertarget{server_daemon_synopsis7}{}\subsection{Synopsis}\label{server_daemon_synopsis7}
{\ttfamily  {\bfseries \#include} {\bfseries $<$\hyperlink{G-2313-06-P1__server_8h}{G-\/2313-\/06-\/\+P1\+\_\+server.\+h}$>$} ~\newline
 {\bfseries void \hyperlink{G-2313-06-P1__server_8c_aa0e8000b12d9c52fc1e87847d00c9c47}{server\+\_\+daemon()}} } \hypertarget{server_daemon_descripcion7}{}\subsection{Descripción}\label{server_daemon_descripcion7}
Crea un nuevo proceso mediante un {\bfseries fork()} y provoca el proceso de damonización por la que el servidor deja de depender de la terminal actual desde la que se ejecuta el servidor y pasa a ejecutarse en segundo plano.\hypertarget{server_daemon_return7}{}\subsection{Valores devueltos}\label{server_daemon_return7}

\begin{DoxyItemize}
\item {\bfseries void} No devuelve nada. 
\end{DoxyItemize}\hypertarget{server_daemon_authors7}{}\subsection{Autores}\label{server_daemon_authors7}

\begin{DoxyItemize}
\item Jorge Parrilla Llamas (\href{mailto:jorge.parrilla@estudiante.uam.es}{\tt jorge.\+parrilla@estudiante.\+uam.\+es}) 
\item Javier de Marco Tomás (\href{mailto:javier.marco@estudiante.uam.es}{\tt javier.\+marco@estudiante.\+uam.\+es}) 
\end{DoxyItemize}\hypertarget{server_start_pool}{}\section{server\+\_\+start\+\_\+pool}\label{server_start_pool}
Inicializa el Thread\+Pool con los hilos.\hypertarget{server_start_pool_synopsis8}{}\subsection{Synopsis}\label{server_start_pool_synopsis8}
{\ttfamily  {\bfseries \#include} {\bfseries $<$\hyperlink{G-2313-06-P1__server_8h}{G-\/2313-\/06-\/\+P1\+\_\+server.\+h}$>$} ~\newline
 {\bfseries void \hyperlink{G-2313-06-P1__server_8c_a48d522cd984dc64ecd084f05416b1a94}{server\+\_\+start\+\_\+pool()}} } \hypertarget{server_start_pool_descripcion8}{}\subsection{Descripción}\label{server_start_pool_descripcion8}
Inicializa el Thread\+Pool con los hilos seleccionados por defecto mediante la variable {\bfseries M\+I\+N\+\_\+\+P\+O\+O\+L\+\_\+\+T\+H\+R\+E\+A\+D\+S}. Para ello ejecuta una llamada la función de la siguiente manera\+:

{\ttfamily  pool = thread\+\_\+pool\+\_\+create(\+M\+I\+N\+\_\+\+P\+O\+O\+L\+\_\+\+T\+H\+R\+E\+A\+D\+S); ~\newline
thread\+\_\+pool\+\_\+init(pool); } \hypertarget{server_start_pool_return8}{}\subsection{Valores devueltos}\label{server_start_pool_return8}

\begin{DoxyItemize}
\item {\bfseries void} No devuelve nada. 
\end{DoxyItemize}\hypertarget{server_start_pool_authors8}{}\subsection{Autores}\label{server_start_pool_authors8}

\begin{DoxyItemize}
\item Jorge Parrilla Llamas (\href{mailto:jorge.parrilla@estudiante.uam.es}{\tt jorge.\+parrilla@estudiante.\+uam.\+es}) 
\item Javier de Marco Tomás (\href{mailto:javier.marco@estudiante.uam.es}{\tt javier.\+marco@estudiante.\+uam.\+es}) 
\end{DoxyItemize}